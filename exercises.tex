\documentclass[a4paper]{article}

\usepackage{times}
\usepackage{listings}
\usepackage{amsmath}

\lstset{basicstyle=\ttfamily}

\setlength{\textwidth}{.75\paperwidth}
\setlength{\textheight}{.9\paperheight}
\hoffset -2.5cm
\voffset -2.5cm
\pagestyle{empty}

\setlength{\parindent}{0cm}
\renewcommand{\baselinestretch}{1.1}


\begin{document}

\section{Exercise}

Implement (type, don't copy-paste), compile and run the following program:

\lstinputlisting[language=C]{solutions/sum/sum.c}


\section{Exercise}

The Fibonacci sequence is given by
\begin{equation*}
F_n = F_{n-1} + F_{n-2}
\end{equation*}
with $F_0 = 0$ and $F_1 = 1$.

Using the summation program as example, implement the Fibonacci sequence calculation
\begin{itemize}
\item Print each term (index and value) inside the loop
\item First, print only the first 5 or 10 numbers to verify the calculation
\item Next, print the first 50 Fibonacci numbers
\end{itemize}

\section{Exercise}

Back to the summation program

\begin{itemize}
\item Wrap the summation in a loop itself, and reset sum inside the outer loop, so it runs longer

\item Find the number of iterations (inner times outer) where the program runs just a few seconds, without optimizations. 
You can use the bash / Linux function time: \lstinline|time <program>|
(Using time is not precise, but a good approximation for a few seconds)

\item Now compile with -O3 (full optimization), and do the same

\item Now use a non-trivial summation, e.g.: \lstinline|sum += log(i);|

  
Use -O3, \lstinline|#include <math.h>| and link with libm (-lm flag)

\item  Now remove the final printf function, and do the same

\end{itemize}

\clearpage

\section{Exercise}

\begin{itemize}
\item Put the Fibonacci calculation inside a function.
Call it from main, and print its return value there

\item Add functionality to return -1 as an error value when the input argument is incorrect (negative or too large)

\item Extra: make a recursive Fibonacci function (remove the for-loop).
  Note: functions calls are expensive: a for-loop is much faster than a series of (recursive) function calls
\end{itemize}

\section{Exercise}

\begin{itemize}

\item Move the Fibonacci function into its own file (or move the main function into its own file): \lstinline|fibo.c| and \lstinline|main.c|
  
\item Create a fibo.h file that has the declaration of the function. Include it in fibo.c and main.c
  
\item Compile \lstinline|fibo.c| and \lstinline|main.c| to objects file (\lstinline|fibo.o| and \lstinline|main.o|): \lstinline|gcc -c <other flags> fibo.c|

  
Link the object files together into an executable
  
\item Turn the \lstinline|fibo.o| object file into a (static) library \lstinline|fibo.a|
  
\item Compile and link \lstinline|main.c| with the \lstinline|fibo.a| in one go
  
\item Extra: create a Makefile to do all the work

\end{itemize}


\section{Exercise}

\begin{itemize}

\item Define a struct Ball that contains an x and y position, and an vx and vy velocity

  \item Instantiate the struct at a global level (outside main and other functions).
Pick some decent initial values;

    \item Create a function that takes as inputs

      - a time interval tdelta

      - a gravity acceleration constant g

      - and a damping factor f
      
      and calculates the next position \& velocity of the ball. The ball bounces when \lstinline|y <= 0|, and loses vy by a factor f (as well as reverses vy).

\end{itemize}

\begin{itemize}

\item In main(), set tdelta, g and f as variables (not constants)

\item Set a variable stop, which is the total time of the simulation

\item Start at t = 0, and loop until t $>$ stop
        
\item Use printf to print the time, x and y position each loop iteration
  
\end{itemize}

\begin{itemize}

\item You can redirect the output to a file: \lstinline|./ball > ballpos.txt|
  
\item You can create a plot of, for example, height versus time in any spreadsheet

\item Or if you like to try gnuplot: \lstinline|gnuplot> plot “ballpos.txt” using 1:3 with lines title “ball”|
  
\end{itemize}

\clearpage

\section{Exercise}

Take the file with the ball main function, and add the following before the loop. 
scanf reads input from the command line (and requires conversion specifiers like printf), and we can ask a user for input.
Note that scanf can’t print output (e.g., a question string). scanf waits for $<$enter$>$ before fully reading the input, and discards whitespace

\begin{lstlisting}
int main()
{
    double g = 9.8, f = 0, tdelta = 1, stop = 1;
    printf(“Value for g: “);
    scanf("%lf", &g);
    printf(“Value for f: “);
    scanf("%lf", &f);
    …
    return 0;
}
\end{lstlisting}

\lstinline|"\%lf"| indicates a double explicitly. Think “long float”.

NB: scanf returns the number of correctly read variables. Use it to verify everything went ok.

\section{Exercise}

Move the global ball variable into the main function. 

Change the function that calculates the ball’s movement to accept a pointer to a Ball struct (as well as the usual g, f, tdelta arguments).

Don’t forget to change the call to the function in main as well.

\section{Exercise}

Store the results of the ball calculation in three arrays: timestamps, xpos and ypos.
You have to calculate the array size from the input time parameters, and dynamically allocate the array (and free the allocation afterwards).

To convert a double to an int, you cast it: 
\begin{lstlisting}
  int a = (int) 5.0;
\end{lstlisting}

Be aware that this truncates the result (that is, rounds down). For example:
\begin{lstlisting}
int a = (int) (10.0/3.0);
a == 3
\end{lstlisting}

Move the printf function out the main loop, and create a second loop afterwards where you only print the results.
(Normally, the calculations would all be done in one function, and printing of the results in another function.)

\clearpage

\section{Exercise}

Change the ball program and compile it with a C++ compiler. You’ll run into a few incompatibilities:

Cast the return value of malloc in C++:
\begin{lstlisting}
  double *timestamps = (double *) malloc(n * sizeof *timestamps); 
\end{lstlisting}

While you won’t get a warning doing this in C, don’t do it: it hides a potential error (forgotten \lstinline|#include <stdlib.h>|).
This is why C++ != C with some extras

C++ does not have designated initializers for structs. Instead, use
\begin{lstlisting}
  struct Ball ball = {0, 10, 1, 0};
\end{lstlisting}
You can do this as well in C, but it is less clear than using the member names.
In C++, structs are quite different beasts, and initialization like this is not really used.

\section{Exercise}

Replace malloc/free with std::vector

\section{Exercise}

Replace printf/scanf with std::cout/std::cin

\section{Exercise}

Replace the Ball struct pointer with a reference

\section{Exercise}

Implement the ball calculation inside the Ball struct

\begin{itemize}

\item use class instead of struct 
\item your choice of private / public 
\item your choice of member variable naming style
\item gravity g and efficiency f should become member variables as well

\end{itemize}

For the constructor: gravity and efficiency should be mandatory arguments
The position and velocity variables should be default arguments.

\subsubsection*{Extra}

\begin{itemize}

\item Store the positions and time in a vector, as member variables of the class

\item Add a member function that takes care of the full loop: it should take a start and stop time, and a timestep. It should call the calculation function each iteration.


  Thus, everything we did in main(), is now done inside the class (apart from the output)

\item Since you’re now not calling the calculation directly, you could make it private.

\item You’ll need to make the relevant output variables public, or define a getter function, if you want to be able to read and print the final results


  Or: you could define a print() or output() class member function for this.
  
\end{itemize}



\end{document}
